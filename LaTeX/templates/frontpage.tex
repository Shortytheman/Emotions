%v Titel
\def\title{Emotions}

% Undertitel
\def\subtitle{Applied Artificial Intelligence}

% Dato
\def\date{June 4, 2024}

% Author
\def\author{
    \textbf{Group: Neon}
    \\
    Emil Breilev Vinther - June 29, 1996 - emil11m8@stud.kea.dk
    \\
    Magnus Büchner Dalkvist - June 6, 2000 - magn552n@stud.kea.dk
    \\
    Naomi Marie Rasmussen - September 12, 1998 - naom0649@stud.kea.dk
    \\
    Niklas Faurholt - August 11, 1989 - nikl5238@stud.kea.dk
}

\def\abstract{
\LARGE\textbf{Abstract}\vskip2mm
\large
This paper explores the prediction and generation of emotions in tweets using machine learning models. The analysis addresses the key research question - Can emotions be predicted from a random tweet? Using the "emotions.csv" dataset\autocite{emotions-dataset}, we used Frequency Distribution Analysis and TF-IDF Analysis to understand the distribution and significance of terms across different emotions.

Our findings reveal a positive sentiment bias, with 'joy' and 'love' being the most prevalent emotions, and 'surprise' significantly underrepresented. TF-IDF analysis identifies key terms indicative of each emotion, while word analysis shows differences in lexical choices across emotions. These insights inform model optimization, including the use of n-grams and balancing techniques to improve accuracy and coherence in emotion prediction.
}

%% FORSIDE %%
\newgeometry{margin=1.8cm}
\thispagestyle{empty}
\begin{flushright}
\includegraphics[width=5cm]{images/kea_logo.jpg}
\end{flushright}	
\vskip30mm
\begin{flushleft}
\large
\huge\textbf{\title}
\vskip2mm
\LARGE\textbf{\subtitle}
\vskip2mm
\large{\date}
\vskip3mm
\author
\end{flushleft}	
\vskip15mm
\abstract
\vfill
\restoregeometry
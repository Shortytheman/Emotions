\section{Conclusion}

Our investigation into emotion prediction within tweets using machine learning techniques provided valuable insights. The predominance of positive sentiments, especially 'joy' and 'love,' suggested potential dataset biases. Despite encountering challenges such as class imbalances and dataset biases, did our models show promise. Notably, the Sequential Model exhibited the highest precision, while Logistic Regression struck a balance between precision and recall. These findings emphasize the potential of machine learning in discerning emotions from social media posts. Addressing dataset biases and class imbalances will be pivotal for refining the accuracy and practicality of emotion prediction models in real-world applications.
So in the end, we found that predicting emotions from tweets proved difficult. When predicting if a person is angry, sad or joyful when writing overly biased answers, the model would respond with the correct prediction most of the time. But the model would never be able to predict the underlying meaning of a message if the user is using sarcasm or for instance is using "double negative" terminology. To train a model with a viable capability of detecting underlying meanings in sentences, we would need a more extensive dataset.